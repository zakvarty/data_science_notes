% Options for packages loaded elsewhere
\PassOptionsToPackage{unicode}{hyperref}
\PassOptionsToPackage{hyphens}{url}
%
\documentclass[
  12pt,
]{book}
\usepackage{amsmath,amssymb}
\usepackage{lmodern}
\usepackage{setspace}
\usepackage{iftex}
\ifPDFTeX
  \usepackage[T1]{fontenc}
  \usepackage[utf8]{inputenc}
  \usepackage{textcomp} % provide euro and other symbols
\else % if luatex or xetex
  \usepackage{unicode-math}
  \defaultfontfeatures{Scale=MatchLowercase}
  \defaultfontfeatures[\rmfamily]{Ligatures=TeX,Scale=1}
\fi
% Use upquote if available, for straight quotes in verbatim environments
\IfFileExists{upquote.sty}{\usepackage{upquote}}{}
\IfFileExists{microtype.sty}{% use microtype if available
  \usepackage[]{microtype}
  \UseMicrotypeSet[protrusion]{basicmath} % disable protrusion for tt fonts
}{}
\makeatletter
\@ifundefined{KOMAClassName}{% if non-KOMA class
  \IfFileExists{parskip.sty}{%
    \usepackage{parskip}
  }{% else
    \setlength{\parindent}{0pt}
    \setlength{\parskip}{6pt plus 2pt minus 1pt}}
}{% if KOMA class
  \KOMAoptions{parskip=half}}
\makeatother
\usepackage{xcolor}
\usepackage{color}
\usepackage{fancyvrb}
\newcommand{\VerbBar}{|}
\newcommand{\VERB}{\Verb[commandchars=\\\{\}]}
\DefineVerbatimEnvironment{Highlighting}{Verbatim}{commandchars=\\\{\}}
% Add ',fontsize=\small' for more characters per line
\usepackage{framed}
\definecolor{shadecolor}{RGB}{248,248,248}
\newenvironment{Shaded}{\begin{snugshade}}{\end{snugshade}}
\newcommand{\AlertTok}[1]{\textcolor[rgb]{0.94,0.16,0.16}{#1}}
\newcommand{\AnnotationTok}[1]{\textcolor[rgb]{0.56,0.35,0.01}{\textbf{\textit{#1}}}}
\newcommand{\AttributeTok}[1]{\textcolor[rgb]{0.77,0.63,0.00}{#1}}
\newcommand{\BaseNTok}[1]{\textcolor[rgb]{0.00,0.00,0.81}{#1}}
\newcommand{\BuiltInTok}[1]{#1}
\newcommand{\CharTok}[1]{\textcolor[rgb]{0.31,0.60,0.02}{#1}}
\newcommand{\CommentTok}[1]{\textcolor[rgb]{0.56,0.35,0.01}{\textit{#1}}}
\newcommand{\CommentVarTok}[1]{\textcolor[rgb]{0.56,0.35,0.01}{\textbf{\textit{#1}}}}
\newcommand{\ConstantTok}[1]{\textcolor[rgb]{0.00,0.00,0.00}{#1}}
\newcommand{\ControlFlowTok}[1]{\textcolor[rgb]{0.13,0.29,0.53}{\textbf{#1}}}
\newcommand{\DataTypeTok}[1]{\textcolor[rgb]{0.13,0.29,0.53}{#1}}
\newcommand{\DecValTok}[1]{\textcolor[rgb]{0.00,0.00,0.81}{#1}}
\newcommand{\DocumentationTok}[1]{\textcolor[rgb]{0.56,0.35,0.01}{\textbf{\textit{#1}}}}
\newcommand{\ErrorTok}[1]{\textcolor[rgb]{0.64,0.00,0.00}{\textbf{#1}}}
\newcommand{\ExtensionTok}[1]{#1}
\newcommand{\FloatTok}[1]{\textcolor[rgb]{0.00,0.00,0.81}{#1}}
\newcommand{\FunctionTok}[1]{\textcolor[rgb]{0.00,0.00,0.00}{#1}}
\newcommand{\ImportTok}[1]{#1}
\newcommand{\InformationTok}[1]{\textcolor[rgb]{0.56,0.35,0.01}{\textbf{\textit{#1}}}}
\newcommand{\KeywordTok}[1]{\textcolor[rgb]{0.13,0.29,0.53}{\textbf{#1}}}
\newcommand{\NormalTok}[1]{#1}
\newcommand{\OperatorTok}[1]{\textcolor[rgb]{0.81,0.36,0.00}{\textbf{#1}}}
\newcommand{\OtherTok}[1]{\textcolor[rgb]{0.56,0.35,0.01}{#1}}
\newcommand{\PreprocessorTok}[1]{\textcolor[rgb]{0.56,0.35,0.01}{\textit{#1}}}
\newcommand{\RegionMarkerTok}[1]{#1}
\newcommand{\SpecialCharTok}[1]{\textcolor[rgb]{0.00,0.00,0.00}{#1}}
\newcommand{\SpecialStringTok}[1]{\textcolor[rgb]{0.31,0.60,0.02}{#1}}
\newcommand{\StringTok}[1]{\textcolor[rgb]{0.31,0.60,0.02}{#1}}
\newcommand{\VariableTok}[1]{\textcolor[rgb]{0.00,0.00,0.00}{#1}}
\newcommand{\VerbatimStringTok}[1]{\textcolor[rgb]{0.31,0.60,0.02}{#1}}
\newcommand{\WarningTok}[1]{\textcolor[rgb]{0.56,0.35,0.01}{\textbf{\textit{#1}}}}
\usepackage{longtable,booktabs,array}
\usepackage{calc} % for calculating minipage widths
% Correct order of tables after \paragraph or \subparagraph
\usepackage{etoolbox}
\makeatletter
\patchcmd\longtable{\par}{\if@noskipsec\mbox{}\fi\par}{}{}
\makeatother
% Allow footnotes in longtable head/foot
\IfFileExists{footnotehyper.sty}{\usepackage{footnotehyper}}{\usepackage{footnote}}
\makesavenoteenv{longtable}
\usepackage{graphicx}
\makeatletter
\def\maxwidth{\ifdim\Gin@nat@width>\linewidth\linewidth\else\Gin@nat@width\fi}
\def\maxheight{\ifdim\Gin@nat@height>\textheight\textheight\else\Gin@nat@height\fi}
\makeatother
% Scale images if necessary, so that they will not overflow the page
% margins by default, and it is still possible to overwrite the defaults
% using explicit options in \includegraphics[width, height, ...]{}
\setkeys{Gin}{width=\maxwidth,height=\maxheight,keepaspectratio}
% Set default figure placement to htbp
\makeatletter
\def\fps@figure{htbp}
\makeatother
\setlength{\emergencystretch}{3em} % prevent overfull lines
\providecommand{\tightlist}{%
  \setlength{\itemsep}{0pt}\setlength{\parskip}{0pt}}
\setcounter{secnumdepth}{5}
\usepackage{booktabs}
\usepackage[inner=30mm,outer=25mm,top=25mm,bottom=25mm]{geometry}
%\usepackage[parfill]{parskip}
\ifLuaTeX
  \usepackage{selnolig}  % disable illegal ligatures
\fi
\usepackage[]{natbib}
\bibliographystyle{apalike}
\IfFileExists{bookmark.sty}{\usepackage{bookmark}}{\usepackage{hyperref}}
\IfFileExists{xurl.sty}{\usepackage{xurl}}{} % add URL line breaks if available
\urlstyle{same} % disable monospaced font for URLs
\hypersetup{
  pdftitle={Effective Data Science},
  pdfauthor={Zak Varty},
  hidelinks,
  pdfcreator={LaTeX via pandoc}}

\title{Effective Data Science}
\author{Zak Varty}
\date{2022-09-20}

\usepackage{amsthm}
\newtheorem{theorem}{Theorem}[chapter]
\newtheorem{lemma}{Lemma}[chapter]
\newtheorem{corollary}{Corollary}[chapter]
\newtheorem{proposition}{Proposition}[chapter]
\newtheorem{conjecture}{Conjecture}[chapter]
\theoremstyle{definition}
\newtheorem{definition}{Definition}[chapter]
\theoremstyle{definition}
\newtheorem{example}{Example}[chapter]
\theoremstyle{definition}
\newtheorem{exercise}{Exercise}[chapter]
\theoremstyle{definition}
\newtheorem{hypothesis}{Hypothesis}[chapter]
\theoremstyle{remark}
\newtheorem*{remark}{Remark}
\newtheorem*{solution}{Solution}
\begin{document}
\maketitle

{
\setcounter{tocdepth}{1}
\tableofcontents
}
\setstretch{1.5}
\hypertarget{preface}{%
\chapter*{Preface}\label{preface}}
\addcontentsline{toc}{chapter}{Preface}

These notes are intended for students on the course \textbf{MATH70076: Data Science} in the academic year 2022/23.

As the course is schedled to take place over five weeks, the suggested schedule is

\begin{itemize}
\tightlist
\item
  1st week: Chapters \ref{intro} and \ref{workflows}
\item
  2nd week: Chapter \ref{data}
\item
  3rd week: Chapter \ref{edav}
\item
  4th week: Chapters \ref{production}
\item
  5th week: Chapter \ref{context}
\end{itemize}

A pdf version of these notes may be downloaded \href{./data_science_notes.pdf}{here}.

\hypertarget{acknowledgements}{%
\section*{Acknowledgements}\label{acknowledgements}}
\addcontentsline{toc}{section}{Acknowledgements}

These notes were created by Dr Zak Varty based on a lecture series at Imperial College London that was developed by Dr Purvasha Chakravarti and others.

\hypertarget{intro}{%
\chapter{Introduction}\label{intro}}

\hypertarget{module-description}{%
\section{Module Description}\label{module-description}}

Model building and evaluation are are necessary but not sufficient skills for the effective practice of data science. In this module you will develop the technical and personal skills that are required to work successfully as a data scientist within an organisation.

During this module you will critically explore how to:

\begin{itemize}
\tightlist
\item
  effectively scope and manage a data science project;
\item
  efficiently acquire, manipulate, and present data;
\item
  interpret and explain your work for a variety of stakeholders;
\item
  ensure that your work can be put into production;
\item
  assess the ethical implications of your work as a data scientist.
\end{itemize}

This interdisciplinary course will draw from fields including statistics, computing, management science and data ethics. Each topic will be investigated through a selection of lecture videos, conference presentations and academic papers, hands-on lab exercises, and readings on industry best-practices from recognised professional bodies.

This course will be assessed entirely by coursework, reflecting the practical and pragmatic nature of the course material.

\hypertarget{allocation-of-study-hours}{%
\section{Allocation of Study Hours}\label{allocation-of-study-hours}}

\textbf{Lectures:} 10 Hours (2 hours per week)

\textbf{Group Teaching:} 5 Hours (1 hour per week)

\textbf{Lab / Practical:} 5 hours (1 hour per week)

\textbf{Independent Study:} 105 hours (17 hours per week + 20 hours coursework)

\hypertarget{learning-outcomes}{%
\subsection{Learning outcomes}\label{learning-outcomes}}

On successful completion of this module students should be able to:

\begin{enumerate}
\def\labelenumi{\arabic{enumi}.}
\tightlist
\item
  Independently scope and manage a data science project;
\item
  Source data from the internet through web scraping and APIs;
\item
  Clean, explore and visualise data, justifying and documenting the decisions made;
\item
  Evaluate the need for (and implement) approaches that are explainable, reproducible and scalable;
\item
  Appraise the ethical implications of a data science projects, particularly the risks of compromising privacy or fairness and the potential to cause harm.
\end{enumerate}

\hypertarget{module-content}{%
\subsection{Module Content}\label{module-content}}

This module will cover:

\begin{itemize}
\tightlist
\item
  effective management of a data science project;
\item
  open and reproducible work flows;
\item
  sourcing and preparing data for analysis;
\item
  exploratory and expository data visualisation;
\item
  minimal requirements for models to go into production;
\item
  ethical implications of modern data science.
\end{itemize}

\hypertarget{an-unnumbered-section}{%
\subsection*{An unnumbered section}\label{an-unnumbered-section}}
\addcontentsline{toc}{subsection}{An unnumbered section}

Chapters and sections are numbered by default. To un-number a heading, add a \texttt{\{.unnumbered\}} or the shorter \texttt{\{-\}} at the end of the heading, like in this section.

\hypertarget{workflows}{%
\chapter{Data Science Workflows}\label{workflows}}

\hypertarget{introduction}{%
\section{Introduction}\label{introduction}}

\begin{itemize}
\item
  As a data scientist you do not work alone.
\item
  Even if you do work alone, you are your own collaborator.
\item
  Treat your future self like a current college who is new to the team
\item
  This is because when you return to the project that you are working on in several weeks, months or years you will have forgotten most of what you did
\item
  You will also have forgotten why you made the decisions that you did and what the other options are.
\item
  The aim here is to provide you with a structure on how you organise and perform your work so that you can be a good collaborator to current collegues and your future self.
\item
  Yes, this is going to require a bit more effort upfront, but the benefits will compound over time.
\item
  The structures and workflows that I recommend will be focused around a workflow that predominantly uses R, markdown and LaTeX. Similar techniques, code and software are available to achieve the same things when working with Python, C, Quarto and a range of other programming and mark-up languages.
\end{itemize}

To motivate this focus, I will rely on an analogy with natural languages. You probably wouldn't want a complex analysis course that tried to teach in English, Japanese and Maori all at once. First you learn the concepts of complex analysis in one language and then (assuming you are already proficient in another) it is a much simpler step to do complex analysis in that language, though it might require you learning some new vocabulary or slightly different syntax.

\hypertarget{organising-and-navigating-your-files}{%
\section{Organising and Navigating your files}\label{organising-and-navigating-your-files}}

\begin{itemize}
\tightlist
\item
  What type of files do you use and where do you code?

  \begin{itemize}
  \tightlist
  \item
    plaintext vs proprietry (csv / xlsx, markdown / googledoc)
  \item
    command line vs notebook vs scripting (no file, .Rmd, .R)
  \item
    Open source languages vs closed source
  \end{itemize}
\item
  All work for one project goes into a single directory

  \begin{itemize}
  \tightlist
  \item
    Portability (filepaths, backslashes, setwd())
  \item
    Version control
  \item
    IDEs play nicely (RStudio projects)
  \item
    reproducibility
  \end{itemize}
\item
  Organising within that directory: Every project different, will develop over course of project. Want to give a sensible starting point but often a company will have a `house style'. If so IGNORE ME (unless the house style is rubbish, in which case only ignore me while you lobby for that to be changed.)

  \begin{itemize}
  \tightlist
  \item
    README.md
  \item
    data

    \begin{itemize}
    \tightlist
    \item
      raw (anything you do not make for yourself)
    \item
      refined (everything that you make for yourself)
    \end{itemize}
  \item
    src (functions)
  \item
    tests (checks for each of your functions)
  \item
    analyses (scripts, models)
  \item
    outputs (results of all your hard work)

    \begin{itemize}
    \tightlist
    \item
      analysis-1

      \begin{itemize}
      \tightlist
      \item
        data
      \item
        tables
      \item
        figures
      \end{itemize}
    \item
      analysis-2

      \begin{itemize}
      \tightlist
      \item
        data
      \item
        tables
      \item
        figures
      \end{itemize}
    \end{itemize}
  \item
    reports (write-up)

    \begin{itemize}
    \tightlist
    \item
      analysis-1
    \item
      analysis-2
    \end{itemize}
  \item
    \emph{bonus:} Makefiles \& meta-programming
  \end{itemize}
\item
  Naming things

  \begin{itemize}
  \item
    Jenny Bryan slide summary (\url{https://speakerdeck.com/jennybc/how-to-name-files})
  \item
    We would like file names to be:

    \begin{itemize}
    \tightlist
    \item
      Machine Readable
    \item
      Human Readable
    \item
      Order friendly
    \end{itemize}
  \item
    Machine readable:

    \begin{itemize}
    \tightlist
    \item
      regex and globbing friendly: avoid spaces, punctuation, accents, cases
    \item
      easy to compute on: deliberate use of delimiters (spaces that aren't spaces)

      \begin{itemize}
      \tightlist
      \item
        hyphens separate words, underscores separate metadata
      \end{itemize}
    \item
      useful when: searching for files later, narrow file list based on names, extract information from file names, new to regex (or not a sadist)
    \end{itemize}
  \item
    Human Readable:

    \begin{itemize}
    \tightlist
    \item
      Name contains information on content. ( untitled31.R, finalreportV8.docx, temp.txt)
    \item
      connects to a concept of a slug from URLs
    \item
      Which set of filenames do you want at 3am before a deadline?
    \end{itemize}
  \item
    Default order friendly

    \begin{itemize}
    \tightlist
    \item
      put something numeric first
    \item
      use ISO 8601 standard for dates: YYYY-MM-DD
    \item
      left pad with zeros to achieve chronological or logical order within each directory
    \end{itemize}
  \item
    Summary:

    \begin{itemize}
    \tightlist
    \item
      Machine readable, human readable, default order friendly
    \item
      Brushing teeth analogy: tedious until you get in the habit. Huge long-term rewards
    \end{itemize}
  \end{itemize}
\item
  Code

  \begin{itemize}
  \tightlist
  \item
    If you do the same thing twice write a function
  \item
    If your write a function, document it
  \item
    If you write a function, test it
  \item
    If might ever want to use your function again, add it to a package
  \item
    naming things revisited:

    \begin{itemize}
    \tightlist
    \item
      functions=verbs,
    \item
      objects=nouns,
    \item
      readable code,
    \item
      CamelCase snakecase pointless.points
    \item
      tidyverse and google style guides for R
    \end{itemize}
  \item
    all filepaths relative to the root directory (the top level of your project)

    \begin{itemize}
    \tightlist
    \item
      advanced: here::here
    \end{itemize}
  \end{itemize}
\item
  Project management

  \begin{itemize}
  \tightlist
  \item
    Defining outcomes
  \item
    scoping projects,
  \item
    continuous development, agile + jira ?
  \item
    Linking to github (extension)
  \end{itemize}
\end{itemize}

Tasks:
- go to github and find 3 different data science projects, explore how they organise their work.
- create your own projects for this course and for the assignments.
- \emph{bonus:} put these on Github (make sure the assignments are private repos!)

Reading:
- Good enough practices in scientific computing
- Bayesian workflows: Micheal Battencourt
- \url{https://www.atlassian.com/agile/project-management}

\begin{itemize}
\tightlist
\item
  R4DS project workflow
\item
  here::here
\item
  R packages
\item
  happy git with R
\end{itemize}

Live session:
Discussion point in live session:
- Did you make the assignment projects as subdirectories or as their stand alone projects? Why?
- What were some terms that you had not met before during the readings?
- Live session activity: making a minimal R package for this course.

\hypertarget{data}{%
\chapter{Aquiring and Sharing Data}\label{data}}

You can add parts to organize one or more book chapters together. Parts can be inserted at the top of an .Rmd file, before the first-level chapter heading in that same file.

Add a numbered part: \texttt{\#\ (PART)\ Act\ one\ \{-\}} (followed by \texttt{\#\ A\ chapter})

Add an unnumbered part: \texttt{\#\ (PART\textbackslash{}*)\ Act\ one\ \{-\}} (followed by \texttt{\#\ A\ chapter})

Add an appendix as a special kind of un-numbered part: \texttt{\#\ (APPENDIX)\ Other\ stuff\ \{-\}} (followed by \texttt{\#\ A\ chapter}). Chapters in an appendix are pre-pended with letters instead of numbers.

\emph{Data can be difficult to acquire and gnarly when you get it.}
- Structure of a webpage. Web Scraping as a source of data
- APIs as a source of data, data files beyond csv.
- Data bases and SQL

\begin{itemize}
\item
  Data will not always be in a nicely formatted csv
\item
  Beginner: reading from clipboard and googlesheets
\item
  Intermediate: reading messy csvs and making your workflow robust to new versions
\item
  Advanced:

  \begin{itemize}
  \tightlist
  \item
    Webscraping and APIs as data sources, data files beyond CSV
  \item
    data files beyond csv, benefits and drawbacks: JSON, xml, parquet, .Rdata, .pkl
  \item
    Data bases as data sources: Basic SQL verbs
  \item
    Learning SQL theory on a small scale: dplyr verbs
  \end{itemize}
\end{itemize}

\hypertarget{edav}{%
\chapter{Cleaning, Exploring and Visualising}\label{edav}}

\hypertarget{footnotes}{%
\section{Footnotes}\label{footnotes}}

Footnotes are put inside the square brackets after a caret \texttt{\^{}{[}{]}}. Like this one \footnote{This is a footnote.}.

\hypertarget{citations}{%
\section{Citations}\label{citations}}

Reference items in your bibliography file(s) using \texttt{@key}.

For example, we are using the \textbf{bookdown} package \citep{R-bookdown} (check out the last code chunk in index.Rmd to see how this citation key was added) in this sample book, which was built on top of R Markdown and \textbf{knitr} \citep{xie2015} (this citation was added manually in an external file book.bib).
Note that the \texttt{.bib} files need to be listed in the index.Rmd with the YAML \texttt{bibliography} key.

The \texttt{bs4\_book} theme makes footnotes appear inline when you click on them. In this example book, we added \texttt{csl:\ chicago-fullnote-bibliography.csl} to the \texttt{index.Rmd} YAML, and include the \texttt{.csl} file. To download a new style, we recommend: \url{https://www.zotero.org/styles/}

The RStudio Visual Markdown Editor can also make it easier to insert citations: \url{https://rstudio.github.io/visual-markdown-editing/\#/citations}

\hypertarget{production}{%
\chapter{Getting Your Work into Production}\label{production}}

\hypertarget{equations}{%
\section{Equations}\label{equations}}

Here is an equation.

\begin{equation} 
  f\left(k\right) = \binom{n}{k} p^k\left(1-p\right)^{n-k}
  \label{eq:binom}
\end{equation}

You may refer to using \texttt{\textbackslash{}@ref(eq:binom)}, like see Equation \eqref{eq:binom}.

\hypertarget{theorems-and-proofs}{%
\section{Theorems and proofs}\label{theorems-and-proofs}}

Labeled theorems can be referenced in text using \texttt{\textbackslash{}@ref(thm:tri)}, for example, check out this smart theorem \ref{thm:tri}.

\begin{theorem}
\protect\hypertarget{thm:tri}{}\label{thm:tri}For a right triangle, if \(c\) denotes the \emph{length} of the hypotenuse
and \(a\) and \(b\) denote the lengths of the \textbf{other} two sides, we have
\[a^2 + b^2 = c^2\]
\end{theorem}

Read more here \url{https://bookdown.org/yihui/bookdown/markdown-extensions-by-bookdown.html}.

\hypertarget{callout-blocks}{%
\section{Callout blocks}\label{callout-blocks}}

The \texttt{bs4\_book} theme also includes special callout blocks, like this \texttt{.rmdnote}.

You can use \textbf{markdown} inside a block.

\begin{Shaded}
\begin{Highlighting}[]
\FunctionTok{head}\NormalTok{(beaver1, }\AttributeTok{n =} \DecValTok{5}\NormalTok{)}
\CommentTok{\#\textgreater{}   day time  temp activ}
\CommentTok{\#\textgreater{} 1 346  840 36.33     0}
\CommentTok{\#\textgreater{} 2 346  850 36.34     0}
\CommentTok{\#\textgreater{} 3 346  900 36.35     0}
\CommentTok{\#\textgreater{} 4 346  910 36.42     0}
\CommentTok{\#\textgreater{} 5 346  920 36.55     0}
\end{Highlighting}
\end{Shaded}

It is up to the user to define the appearance of these blocks for LaTeX output.

You may also use: \texttt{.rmdcaution}, \texttt{.rmdimportant}, \texttt{.rmdtip}, or \texttt{.rmdwarning} as the block name.

The R Markdown Cookbook provides more help on how to use custom blocks to design your own callouts: \url{https://bookdown.org/yihui/rmarkdown-cookbook/custom-blocks.html}

\hypertarget{context}{%
\chapter{Wider Context and Ethics}\label{context}}

\hypertarget{publishing}{%
\section{Publishing}\label{publishing}}

HTML books can be published online, see: \url{https://bookdown.org/yihui/bookdown/publishing.html}

\hypertarget{pages}{%
\section{404 pages}\label{pages}}

By default, users will be directed to a 404 page if they try to access a webpage that cannot be found. If you'd like to customize your 404 page instead of using the default, you may add either a \texttt{\_404.Rmd} or \texttt{\_404.md} file to your project root and use code and/or Markdown syntax.

\hypertarget{metadata-for-sharing}{%
\section{Metadata for sharing}\label{metadata-for-sharing}}

Bookdown HTML books will provide HTML metadata for social sharing on platforms like Twitter, Facebook, and LinkedIn, using information you provide in the \texttt{index.Rmd} YAML. To setup, set the \texttt{url} for your book and the path to your \texttt{cover-image} file. Your book's \texttt{title} and \texttt{description} are also used.

This \texttt{bs4\_book} provides enhanced metadata for social sharing, so that each chapter shared will have a unique description, auto-generated based on the content.

Specify your book's source repository on GitHub as the \texttt{repo} in the \texttt{\_output.yml} file, which allows users to view each chapter's source file or suggest an edit. Read more about the features of this output format here:

\url{https://pkgs.rstudio.com/bookdown/reference/bs4_book.html}

Or use:

\begin{Shaded}
\begin{Highlighting}[]
\NormalTok{?bookdown}\SpecialCharTok{::}\NormalTok{bs4\_book}
\end{Highlighting}
\end{Shaded}


  \bibliography{book.bib,packages.bib}

\end{document}
